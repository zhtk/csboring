% Preambuła
\documentclass[a4paper,11pt]{article}
\usepackage[polish]{babel}
\usepackage[OT4]{fontenc}
\usepackage[utf8]{inputenc}
\usepackage{geometry}
\usepackage{fancyhdr}
\usepackage{minted}
\usepackage{graphicx}

\geometry{lmargin=2cm}
\pagestyle{fancy}
\lhead{ZBD - zadanie 3}
\rhead{Piotr Zalas}

% Część główna
\begin{document}
Na początku połączyłem się z bazą jako \textit{system/manager} i wykonałem polecenia tworzące
perspektywę i partycje:
\begin{minted}{sql}
ALTER SESSION SET CURRENT_SCHEMA = SH;
@rdbms/admin/utlxrw.sql
@costs_promo_mv.sql
@partitions.sql
\end{minted}
Następnie sprawdziłem, czy wszystko działa poprzez wykonanie zapytania
\begin{minted}{sql}
SELECT p.promo_subcategory, sum(unit_cost)
FROM costs c, promotions p
WHERE c.promo_id = p.promo_id
GROUP BY promo_subcategory;
\end{minted}
Teraz już mogłem przystąpić do sprawdzenia, na co zostało przepisane to zapytanie. Zrobiłem to w ten sposób:
\begin{minted}{sql}
BEGIN
DBMS_MVIEW.EXPLAIN_REWRITE(
'SELECT p.promo_subcategory, sum(unit_cost)
FROM costs c, promotions p
WHERE c.promo_id = p.promo_id
GROUP BY promo_subcategory',
'SH.COSTS_PROMO_MV',
'Z1');
END;
/
SELECT message FROM REWRITE_TABLE;
SELECT rewritten_txt FROM REWRITE_TABLE;
\end{minted}
Naturalnie musiałem wykonać to polecenie jako użytkownik \textit{SH}, administrator bazy danych na koncie \textit{system} nie ma dostępu do tak zaawansowanych i potencjalnie niebezpiecznych funkcji, które dla podniesienia poziomu bezpieczeństwa wypluwają mu enigmatyczny komunikat o nieistnieniu tabeli... W wyniku otrzymałem optymistycznie brzmiący komunikat:
\begin{minted}{text}
MESSAGE
-----------------------------------------------------------------
QSM-01151: query was rewritten
QSM-01033: query rewritten with materialized view, COSTS_PROMO_MV

REWRITTEN_TXT
-----------------------------------------------------------------
SELECT COSTS_PROMO_MV.PROMO_SUBCATEGORY PROMO_SUBCATEGORY,
SUM(COSTS_PROMO_MV.SUM_UNITS) SUM(UNIT_COST) 
FROM SH.COSTS_PROMO_MV COSTS_PROMO_MV
GROUP BY COSTS_PROMO_MV.PROMO_SUBCATEGORY
\end{minted}
Tytułem wyjaśnienia, nasz widok przechowuje zagregowane koszty jednostkowe z poszczególnych dni. Nie ma powodu, by przy wykonaniu zapytania znowu sumować wszystkie pojedyńcze koszty, skoro można się mniej napracować i zsumować już gotowe wyniki z kolejnych dni. Zapewne optymalizator zastosował takie rozumowanie przy układaniu planu.
Zgodnie z kolejnym poleceniem usunąłem wszystkie wiersze z partycji COST\_Q1\_2004. Usunięto ok. 82 tysięcy wierszy. Od tej chwili rewrite nie następuje z uwagi na wykonanie operacji DML na tabeli:
\begin{minted}{text}
SM-01150: query did not rewrite
QSM-01279: query rewrite not possible because DML operation occurred on a table 
referenced by materialized view COSTS_PROMO_MV
\end{minted}
Naprawienie tej sytuacji jest proste - trzeba zaktualizować widok. Aby taka sytuacja nie zachodziła w przyszłości warto zmienić tryb odświeżania widoku z ,,REFRESH FAST ON DEMAND'' na ,,REFRESH FAST ON COMMIT''. Ja ograniczyłem się do wywołania
\begin{minted}{sql}
EXEC DBMS_MVIEW.refresh('COSTS_PROMO_MV', 'f')
\end{minted}
Wykonanie EXPLAIN\_REWRITE potwierdziło, że wszystko znowu działa. Mogłem przejść do dalszej części zadania:
\begin{minted}{sql}
ALTER SESSION SET QUERY_REWRITE_INTEGRITY = TRUSTED;
EXEC DBMS_MVIEW.EXPLAIN_REWRITE(
'SELECT p.promo_category, sum(unit_cost)
FROM costs c, promotions p
WHERE c.promo_id = p.promo_id
AND c.time_id >= TO_DATE('''01-JAN-2000''','''DD-MON-YYYY''')
AND c.time_id < TO_DATE('''01-JAN-2001''','''DD-MON-YYYY''')
GROUP BY promo_category',
'SH.COSTS_PROMO_MV',
'Z2')
SELECT message FROM REWRITE_TABLE WHERE statement_id = 'Z2';
\end{minted}
Przepisanie nie zachodzi, wyjaśnienie jest takie:
\begin{minted}{text}
QSM-01150: query did not rewrite                                                
QSM-01082: Joining materialized view, COSTS_PROMO_MV, with table, PROMOTIONS,
not possible
QSM-01102: materialized view, COSTS_PROMO_MV, requires join back to table, 
PROMOTIONS, on column, PROMO_CATEGORY
QSM-01053: NORELY referential integrity constraint on table, PROMOTIONS, in 
TRUSTED/STALE TOLERATED integrity mode
\end{minted}
Pierwszym zasadniczym pytaniem jest, czemu tutaj następuje jakikolwiek join? Przecież wszystkie potrzebne informacje
są zawarte w widoku COSTS\_PROMO\_MV! Po dokładniejszym przyjrzeniu się definicjom obiektów okazuje się, że tabela
PROMOTIONS jest potrzebna do konsolidacji podkategorii w kategorie. Baza danych ma mgliste pojęcie o zależności między kategoriami i podkategoriami dzięki informacji zawartej w obiekcie PROMOTIONS\_DIM, jednak z widoku nie może wyczytać konkretnych wartości. Postanowiłem odrobinę wyklarować sytuację poprzez utworzenie kolejnego obiektu:
\begin{minted}{sql}
CREATE DIMENSION dim_specjal
  LEVEL subcategory IS (promotions.promo_subcategory) 
  LEVEL category IS (promotions.promo_category) 
  HIERARCHY roll (subcategory CHILD OF category)
\end{minted}
Wygenerowało to następujący efekt:
\begin{minted}{text}
QSM-01102: materialized view, COSTS_PROMO_MV, requires join back to table, 
PROMOTIONS, on column, PROMO_CATEGORY
QSM-01151: query was rewritten
QSM-01110: query rewrite not possible with materialized view COSTS_PROMO_MV 
because it contains a join between tables (PROMOTIONS and COSTS) that is not present
in the query and that potentially eliminates rows needed by the query
QSM-01033: query rewritten with materialized view, COSTS_PROMO_MV
\end{minted}
Jak widać mamy demokrację, planer jest za, a nawet przeciw. Jestem skłonny uznać, że stosunkiem 2:1 wygrała opcja
mówiąca, że rewrite jednak się odbyło. Świadczą o tym wygenerowane zapytania:
\begin{minted}{text}
SELECT DISTINCT PROMOTIONS.PROMO_SUBCATEGORY PROMO_SUBCATEGORY,PROMOTIONS.PROMO_
CATEGORY PROMO_CATEGORY FROM PROMOTIONS PROMOTIONS GROUP BY PROMOTIONS.PROMO_SUB
CATEGORY,PROMOTIONS.PROMO_CATEGORY                                              

SELECT from$_subquery$_005.PROMO_CATEGORY PROMO_CATEGORY,SUM(COSTS_PROMO_MV.SUM_
UNITS) SUM(UNIT_COST) FROM SH.COSTS_PROMO_MV COSTS_PROMO_MV, (SELECT DISTINCT PR
OMOTIONS.PROMO_SUBCATEGORY PROMO_SUBCATEGORY,PROMOTIONS.PROMO_CATEGORY PROMO_CAT
EGORY FROM PROMOTIONS PROMOTIONS) from$_subquery$_005 WHERE from$_subquery$_005.
PROMO_SUBCATEGORY=COSTS_PROMO_MV.PROMO_SUBCATEGORY AND COSTS_PROMO_MV.TIME_ID>=T
O_DATE("01-JAN-2000","DD-MON-YYYY") AND COSTS_PROMO_MV.TIME_ID<TO_DATE("01-JAN-2
001","DD-MON-YYYY") GROUP BY from$_subquery$_005.PROMO_CATEGORY
\end{minted}
Planer (albo ,,przepisywaczka zapytań'') wydedukował sposób generacji kategorii z subkategorii, a potem użył go do przepisania. W moim odczuciu wygenerowane zapytanie jest całkiem sensowne i jest cień szansy, że realizuje zamierzenie pierwotnego zapytania. Innym sposobem na osiągnięcie takiego efektu byłoby wprowadzenie równoważności między id kategorii (subkategorii), a jej nazwą poprzez wprowadzenie dodatkowych dwóch atrybutów do PROMOTIONS\_DIM. Zrezygnowałem z takiego rozwiązania, gdyż o ile przy danych w przykładowej bazie danych jest to prawdą, o tyle w ogólności takie zjawisko może nie mieć miejsca.
\end{document}

